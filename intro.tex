\lecture{Fundamentos}{foundation}

\title{\course \bf\insertlecture}

\frame{\maketitle}

\section{\insertlecture}

\begin{frame}{Principais Tópicos}

\begin{center}
  \begin{tikzpicture}[small mindmap]
    \begin{scope}[
      every node/.style={concept, circular drop shadow,execute at begin node=\hskip0pt},
      root concept/.append style={concept color=black, fill=white, line width=1ex, text=black, font=\footnotesize\scshape},
      text=white,
      logic/.style={concept color=red,faded/.style={concept color=red!50}},
      number theory/.style={concept color=blue,faded/.style={concept color=blue!50}},
      graph theory/.style={concept color=orange,faded/.style={concept color=orange!50}},
      combinatorics/.style={concept color=green!50!black,faded/.style={concept color=green!50!black!50}},
      set theory/.style={concept color=black!80,faded/.style={concept color=yellow!50!black!50}},
      grow cyclic,
      level 1/.append style={level distance=2.5cm,sibling angle=72,font=\scriptsize},
      level 2/.append style={level distance=1.75cm,sibling angle=45,font=\tiny}]
      \node [root concept] {Matemática Discreta} % root
      child [logic] { node {Lógica}
        child { node {Proposições} }
        child { node {Provas} }
      }
      child [graph theory,faded] { node {Teoria dos Grafos}
      }
      child [set theory] { node {Teoria dos Conjuntos}
        child { node {Cardinalidade} }
        child { node {Junções} }
        child { node {Relações} }
      }
      child [number theory] { node {Teoria dos Números}
        child { node  {MDC} }
        child { node  {MMC} }
        child { node (primes) {Números Primos} }
        child { node {Módulo} }
      }
      child [combinatorics] { node {Análise Combinatória}
        child { node {Arranjos} }
        child { node {Permutações} }
        child { node {Combinações} }
      }
      node (recursion) [extra concept,right of=primes,xshift=4cm] {Recursão}
      node [extra concept,below of=recursion,yshift=-1cm] {Indução};

    \end{scope}
    \end{tikzpicture}
  \end{center}

\end{frame}

\begin{frame}{Definição}

  A {\bf Matemática Discreta} manipula objetos \alert{contáveis} utilizando
  sistemas matemáticos adequados para cada tipo de problema. 
  Exemplo: números inteiros.
  
\end{frame}

\begin{frame}{Algumas aplicações da Matemática Discreta}

  \begin{itemize}[<+-| alert@+>]

  \item Métodos formais para especificação, verificação e validação de
    software;
  \item Análise da complexidade dos algoritmos;
  \item Algoritmos combinatórios: grafos, expressões booleanas, $\dots$;
  \item Probablidade;
  \item Computabilidade;
  \item Cálculo numérico;
  \item Criptografia
  \item $\dots$
  \end{itemize}
  
\end{frame}

\section{Noções de Lógica}

\frame{\author{}\date{}\title{Noções de Lógica}\maketitle}

\subsection{Axioma}

\begin{frame}{Axioma ou Postulado}

\alert{Axioma} é uma premissa ou ponto de partida para o raciocínio. 
É tão evidente que é aceito como verdadeiro sem controvésia e sem necessidade 
de prova.

\bigskip
\onslide<2>
Exemplo: Axioma de Peano para as propriedades dos números naturais.

\begin{enumerate}
\item $0 \in Nat$
\item $x=x$
\item $x=y \Rightarrow y=x$
\item $x=y\wedge y=z\Rightarrow x=z$
\end{enumerate}
  
\end{frame}

\subsection{Teorema}

\begin{frame}{Teorema}

\alert{Teorema} é uma afirmação cuja veracidade pode ser demonstrada 
utilizando operações e argumentos matemáticos aceitáveis.

\bigskip
\onslide<2>
Exemplo: Teorema de Pitágoras\\

\begin{center}
\begin{tikzpicture}
  \node at (-1,2) {$c^2=a^2+b^2$};
  \path[draw] (0,0) -- (2,0) node[anchor=north west] {$a$} -- (4,0) -- 
  (4,1.5) node[anchor=west] {$b$} --  (4,3) -- (2,1.5) node[anchor=south] {$c$} --cycle;
\end{tikzpicture}
\end{center}

\end{frame}

\subsection{+Termos}
\begin{frame}{+Termos}
  
  \alert{Definição} especifica o significado de palavra ou frase em utilizando
  conceitos já conhecidos. Também é aceita sem prova.\\
  \bigskip
  \pause
  \alert{Proposição} é um termo genérico para um teorema sem importância particular. 
  Este termo denota um afirmação que exige um prova simples, enquanto {\bf teorema} é 
  reservado para resultados de maior importância com prova difícil ou longa.\\
  \bigskip
  \pause
  \alert{Lema} é uma {\bf proposição} que faz parte da prova de um teorema por possuir 
  pouca aplicabilidade.\\
  \bigskip
  \pause
  \alert{Corolário} é uma {\bf proposição} que segue com pouca ou nenhuma prova outra 
  definição ou teorema.\\
\end{frame}

\end{document}



