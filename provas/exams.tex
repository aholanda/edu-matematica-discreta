\documentclass[11pt, a4paper]{article}

\usepackage[]{fontspec}
\usepackage[brazil]{babel}
\usepackage{enumerate}
\usepackage[right=2cm,bottom=2cm]{geometry}

\title{Exame de Matemática Discreta}
\author{prof. Adriano J. Holanda}
\date{\today}

\def\tupla#1{\langle #1 \rangle}

\begin{document}

\maketitle

\noindent Faculdade ``Dr. Francisco Maeda'' -- FAFRAM\\
\noindent Curso: Sistemas de Informação, Ciclo: $4^o$\\
\noindent Nome completo:

\paragraph{Questão 0 (3 pontos)} Determinar se as seguintes relações
são: reflexiva, antirreflexiva, simétrica e antisimétrica:

\begin{enumerate}[(a)]
\item
$R_1=\{\tupla{1,3},\tupla{1,1},\tupla{2,2},\tupla{3,1},\tupla{3,3}\}$
\item
$R_2=\{\tupla{1,1},\tupla{2,2},\tupla{3,3},\tupla{1,2},\tupla{2,3}\}$
\item $R_3=\{\tupla{1,1},\tupla{1,2},\tupla{2,3},\tupla{3,1}\}$
\item $R_4 = A\times A$
\item $R_5 = \emptyset$
\end{enumerate}

\noindent Explique os critérios utilizados para a classificação das relações.

\paragraph{Questão 1. (3 pontos)} Ache o máximo divisor comum (MDC), demonstrando os passos
intermediários das operações, utilizando o algoritmo de Euclides para
os seguintes valores:

\begin{enumerate}[(a)]
\begin{minipage}{.5\textwidth}
\item mdc(15, 5)
\item mdc(23, 19)
\item mdc(12, 15)
\end{minipage}
\begin{minipage}{.5\textwidth}
\item mdc(21, 18)
\item mdc(21, 19)
\item mdc(31, 46)
\end{minipage}
\end{enumerate}

\paragraph{Questão 2.(2 ponto(s))} Quantos números de 3 algarismos
podemos formar com os dígitos 2, 3, 4, 5, 8, 9, supondo que não haja
repetição de dígitos nos números formados.

\paragraph{Questão 3.(2 ponto(s))} Quantos números de 3 algarismos
podemos formar com os dígitos primos $p$, com $2\leq p \leq 19$, supondo
que não haja repetição de dígitos nos números formados.


\vfill

\hrule\bigskip
{\footnotesize
\noindent {\bf Regras}\\
\begin{enumerate}
\item Duração da prova: 1 hora;
\item Prova sem consulta;
\item A prova pode ser feita a lápis;
\item Não é permitido usar calculadora.

\end{enumerate}
}

\begin{flushright}
\bf\large  Boa Prova
\end{flushright}

\end{document}

%%%%%%%%%%%%%%%%%%%%%%%%%%%%%%%% BUFFER
 \paragraph{Questão 0. (4 pontos)} Sejam $A=\{2,3,4,5,12\}$ e $B=\{3,4,6,10\}$. Para cada uma
 das seguintes relações:
 \label{ex:rel}

 \begin{enumerate}[(a)]
 \item $R_1=\{\langle x,y\rangle\in A\times B : x \mbox{ é divisível
 por } y\}$
 \item $R_2=\{\langle x,y\rangle\in A\times B : x \mbox{ é ímpar } \land
 y \mbox{ é par}\}$
 \item $R_3=\{\langle x,y\rangle\in A\times B : x \mbox{ e } y \mbox{
 sejam números primos}\}$
% \item $R_4=\{\langle x,y\rangle\in A\times B : y=2*x\}$
 \item $R_4=\{\langle x,y\rangle\in A\times B : x=y \lor
   y=x+1\}$~\footnote{$\lor$ - conectivo lógico {\tt ou}.}
 \end{enumerate}

 \begin{enumerate}[(1)]
 \item explicite os pares ordenados da relação;
 \item faça a representação gráfica;
 \item determine o domínio da definição;
 \item determine o conjunto imagem.
 \end{enumerate}



\paragraph{Questão 2. (3 pontos)} Efetue as operações {\tt módulo} ($\Delta$) a seguir, mostrando
os cálculos intermediários:
\def\mod#1#2{$#1\Delta#2$}
\begin{enumerate}[(a)]
\begin{minipage}{.5\textwidth}
\item \mod{23}{5}
\item \mod{23}{-5}
\item \mod{-23}{5}
\end{minipage}
\begin{minipage}{.5\textwidth}
\item \mod{-23}{-5}
\item \mod{555}{444}
\item \mod{1000}{323}
\end{minipage}
\end{enumerate}

 
%%%%%%%%%%%%%%%%%%%%%%%%%%%%%%%% BEGIN OF TESTE 1

\ifnum\exam=\testeum

\exnew{3} Dados os $A=\{1\}$; $B=\{2\}$ e $C = \{\{1\}, 1\}$, classifique
como verdadeira (V) ou falsa (F) as seguintes afirmações:

\bigskip

\begin{minipage}{0.45\textwidth}
\begin{enumerate}[a. (\ )]
\item $A\subset B$ %X
\item $A\subseteq B$%X
\item $A = B$
\item $A \in B$
\item $A\subset C$%X
\item $A\subseteq C$%X
\item $A\in C$%X
\end{enumerate}
\end{minipage}
\begin{minipage}{0.45\textwidth}
\begin{enumerate}[A. (\ )]
\item $A = C$
\item $1 \in A$%X
\item $1 \in C$%X
\item $\{1\} \in A$
\item $\{1\} \in C$%X
\item $\emptyset \notin C$%X
\item $\emptyset \subseteq C$%X
\end{enumerate}
\end{minipage}

\exnew{3} Dados os conjuntos $A=\{1,2,3\}$, $B=\{3,4,5\}$ e
$C=\{4,5,7\}$, classifique como verdadeira (V) ou falsa (F) as seguintes
afirmações:

\begin{enumerate}[(a) ( )]
\item $A\cup B = \{1,2,4,5\}$
\item $A\cap B = \{3,7\}$
\item $A\cap C = \{7\}$
\item $A\cap B\cap C = \emptyset$
\item $A - B = \{1,2\}$
\item $B - A = \{4,5\}$
\item $A\Delta B = \{1,2,4,5\}$
\item $A\cup B \cap C = \{4,5\}$
\item $A\cap B \cup C = \{4,5,7\}$
\item $ B \Delta C = \{3,7\}$
\end{enumerate}

\exnew{4} Dados os conjuntos $A=\{2,3,6,7,8,11\}$, $B=\{0,1,4,5,8,9,12,15\}$, classifique
como verdadeira (V) ou falsa (F) as seguintes afirmações:

\def\tuple#1{\langle{#1}\rangle}
\begin{enumerate}[(a)( )]
\item $R_1=\{\tuple{x,y}\in A\times B:x=\sqrt{y}\}=\{\tuple{2,4},
  \tuple{3,9}\}$
\item $R_2=\{\tuple{x,y}\in A\times B:x=y^2\}=\{\tuple{2,4}, 
  \tuple{3,9}\}$
\item $R_3=\{\tuple{x,y}\in A\times B:x=y+1\}=\{\tuple{3,4},\tuple{7,8},
  \tuple{8,9}\}, \tuple{11,12}$
 \item $R_4=\{\tuple{x,y}\in A\times B:x \mbox{ é par e primo } \land
  y\mbox{ é impar e primo }\}=\{\tuple{2,1}, \tuple{2,5}, \tuple{3,1},
  \tuple{3,5}, \tuple{7,1}, \tuple{7,15}\}$
 \item $R_5=\{\tuple{x,y}\in A\times B:x=y-4\}=\{\tuple{8,12},
   \tuple{11,15}\}$
\item $R_6=\{\tuple{x,y}\in A\times B:x \mbox{ é divisível por }
  y\}=\{\tuple{8,4}, \tuple{8,8}\}$
\item $R_7=\{\tuple{x,y}\in A\times B:y=x*2\}=\{\tuple{2,4},
\tuple{6,12}, \tuple{8,16}\}$
\item $R_8=\{\tuple{x,y}\in A\times B:x\mbox{ é par } \land y \mbox{ é
    negativo }\}=\emptyset$
\item $R_9=\{\tuple{x,y}\in A\times B:x=\sqrt[3]{y}\}=\{\tuple{2,8}\}$
\item $R_{10}=\{\tuple{x,y}\in A\times B:x=y^3\}=\{\tuple{2,8}\}$
\end{enumerate}

\fi

%%%%%%%%%%%%%%%%%%%%%%%%%%%%%%%% END OF TESTE 1


%%%%%%%%%%%%%%%%%%%%%%%%%%%%%%%% BEGIN OF PROVA 1

\ifnum\exam=\provaum



%%%%%%%%%%%%%%%%%%%%%%%%%%%%%%%% END OF PROVA 1
