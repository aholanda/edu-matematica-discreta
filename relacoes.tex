\lecturetitle{\course}{Relações e Cardinalidade de Conjuntos}

\frame{\maketitle}

\lecture{Relações}{relations}

\section{\inserttitle}

\begin{frame}{\inserttitle}
\small
  \begin{block}{Definição}
    Sejam $A$ e $B$ conjuntos. Uma \alert{relação} $R$ de $A$ em $B$ é
    um subconjunto de um produto cartesiano $A\times B$, ou seja:
    \[R=A\subseteq B\]
    \noindent sendo que:\\
    \noindent $A$ é denominado \emph{domínio} de $R$ e\\
    \noindent $B$ \emph{contradomínio} de $R$.
  \end{block}
  \bigskip\onslide<2>
Uma relação $R\subseteq A\times B$ também é denotada como segue:
\[ R:A\rightarrow B\]
\noindent e um elemento $\tu{a,b}\in R$ é frequentemente denotao de
forma infixada, como segue:
\[a\ R\ b\]

\end{frame}

\begin{frame}{Relação}
  
  \begin{block}{Exemplo}
    Sejam $A=\{a\}$, $B=\{a,b\}$ e $C=\{0,1,2\}$. Então, são relações:
    \begin{enumerate}[<+-| alert@+>]
    \item $\emptyset$ é uma relação de $A$ em $B$, $A$ em $C$ e $B$ em $C$;
    \item $A\times B=\{\tu{a,a},\tu{a,b}\}$ é uma relação com origem
      em $A$ e destino em $B$;
    \item Considerando o domínio $A$ e o contradomínio $B$, a
      \emph{relação} de igualdade é $\{\tu{a,a}\}$;
    \item $\{\tu{0,1}, \tu{0,2}, \tu{1,2}\}$ é a relação ``menor que''
      ($<$) de $C$ em $C$;
    \item $\{\tu{0,a},\tu{1,b}\}$ é uma relação de $C$ em $B$.
    \end{enumerate}
  \end{block}

\end{frame}

\def\mysection{Domínio de definição, conjunto imagem}

\subsection{\mysection}

\begin{frame}{\mysection}
  
  \begin{block}{Definição}
    Seja {\bf $R:A\rightarrow B$} uma relação. Então:

    \begin{itemize}[<+-| alert@+>]
    \item Se $\tu{a,b}\in R$, então se afirma que $R$ está \emph{definida}
      para $A$, e que $b$ é a \emph{imagem} de $a$;
    \item O conjunto de todos os elementos de $A$ para as quais $R$
      está definida é denominada \emph{domínio de definição};
    \item O conjunto de todos os elementos de $B$, imagem de $R$, é
      denominada \emph{conjunto imagem}.
    \end{itemize}
  \end{block}
  
\end{frame}

\begin{frame}{\mysection}

  \begin{block}{Exemplo}
    Sejam $A=\{a\}$, $B=\{a,b\}$ e $C=\{0,1,2\}$, então:
    
    \begin{enumerate}[<+-| alert@+>]
    \item Para a relação $\emptyset$: $A\rightarrow B$, o domínio de
      definição e o conjunto imagem são vazios;
    \item Para a \alert{endorrelação$^*$} $\tu{C,<}$, dado que $<$ é
      definida por $\{\tu{0,1}, \tu{0,2}, \tu{1,2}\}$, o domínio de
      definição é $\{0,1\}$ e a imagem é $\{1,2\}$;
    \item Para a relação $=:A\rightarrow B$, o conjunto $\{a\}$ é o
      domínio de definição e o conjunto imagem.
    \end{enumerate}
  \end{block}

\hfill
$^*$ Endorrelação ou autorrelação é uma relação $R:A\rightarrow A$
(origem e destino no mesmo conjunto).
  
\end{frame}

\def\mysection{Aplicações}
\subsection{\mysection}

\begin{frame}{\mysection}
\footnotesize
  \begin{block}{Banco de Dados Relacional}
    Sejam os conjuntos \emph{País} e \emph{Continente}, aplicando a
    relação \emph{``Fica em''} obtemos: \\
    \bigskip
    \begin{minipage}[h]{.25\linewidth}
      \begin{tabular}[h]{|c|}\hline
        \bf País \\\hline
        Brasil\\
        Turquia\\
        Alemanha\\
        Coreia do Sul\\
        Austrália\\\hline
      \end{tabular}
    \end{minipage}
    \begin{minipage}[h]{.25\linewidth}
    \begin{tabular}[h]{|c|}\hline
      \bf Continente\\\hline
      América\\
      Oceania\\
      África\\
      Ásia\\
      Europa\\\hline
    \end{tabular}\onslide<2>
    \end{minipage}$\Rightarrow$
    \begin{minipage}[h]{.35\linewidth}
      \begin{tabular}[h]{|c|c|}\hline
        \multicolumn{2}{|c|}{\bf Fica em} \\\hline
        Brasil & América\\
        Turquia & Europa\\
        Turquia & Ásia\\
        Alemana & Europa\\
        Coreia do Sul & Ásia\\
        Austrália & Oceania\\\hline
      \end{tabular}
    \end{minipage}
  \end{block}

\end{frame}

\def\mysection{Exercício}
\section{\mysection}
\begin{frame}{\mysection}
  \small
  Sejam $A=\{2,3,4,5\}$ e $B=\{3,4,5,6,10\}$, para cada uma das
  seguintes relações:
  \begin{enumerate}\footnotesize
    \begin{minipage}[h]{.475\linewidth}
    \item $R_1=\{\tu{x,y}\in A\times B|$\\$x\text{ é divisível por }y\}$\\
    \item $R_2=\{\tu{x,y}\in A\times B$\\$|x*y=12\}$\\
    \end{minipage}
    \begin{minipage}[h]{.45\linewidth}
    \item $R_3=\{\tu{x,y}\in A\times B|x=y+1\}$\\
    \item $R_3=\{\tu{x,y}\in A\times B|x\geq y\}$\\
    \end{minipage}
  \end{enumerate}
  \begin{itemize}
  \item explicite as tuplas da relação;
  \item faça uma representação gráfica no plano cartesiano;
  \item determine o domínio de definição;
  \item determine o conjunto imagem.
  \end{itemize}
\end{frame}

\begin{frame}{Bibliografia}
  \begin{thebibliography}{5}
  \bibitem[Menezes]{menezes2010}
    {Paulo Blauth Menezes}
    \newblock {Matematica Discreta para Computa\c{c}\~{a}o e Inform\'{a}tica}
    \newblock {Editora Bookman}, {2010}, {3$^a$ edi\c{c}\~{a}o}.
  \end{thebibliography}
\end{frame}

