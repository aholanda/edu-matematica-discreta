\section{Relações}

    \textbf{Alguns conjuntos importantes}

\(\mathbb{B}\) = valores Boolean = \(\{true, false\}\)

\(\mathbb{N}\) = números naturais = \{0, 1, 2, 3, . . . \}

\(\mathbb{Z}\) = inteiros = \{. . . , −3, −2, −1, 0, 1, 2, 3, . . . \}

\(\mathbb{Z}^+\) = \(\mathbb{Z}_{≥1}\) = inteiros positivos = \{1, 2, 3,
. . . \}

\(\mathbb{R}\) = conjunto dos números reais

\(\mathbb{R}^+\) = \(\mathbb{R}_{>0}\) = conjunto dos números reais
positivos

\(\mathbb{C}\) = conjunto dos números complexos

\(\mathbb{Q}\) = conjunto dos números racionais

    \hypertarget{definiuxe7uxe3o}{%
\subsection{Definição}\label{definiuxe7uxe3o}}

Sejam \(A\) e \(B\) conjuntos. Uma \textbf{relação} \(R\) de \(A\) em
\(B\) é um subconjunto de produto cartesiano \(A\times B\), ou seja:

\[R=A\subseteq B\]

sendo que:

\(A\) é denominado \emph{domínio} de \(R\) e

\(B\) \emph{contradomínio} de \(R\).

    Uma \textbf{relação} \(R\subseteq A \times B\) também é denotada como

\[R: A\rightarrow B\]

e um elemento \((a,b)\in R\) é frequentemente denotado de forma infixada
como

\[a\ R\ b\]

    \hypertarget{exemplo}{%
\subsubsection{Exemplo}\label{exemplo}}

Sejam \(A=\{a\}\), \(B=\{a,b\}\) e \(C=\{0,1,2\}\). Então são relações:

    \begin{enumerate}
\def\labelenumi{\arabic{enumi})}
\item
  \(\emptyset\) é uma \textbf{relação} de \(A\) em \(B\), \(A\) em \(C\)
  e \(B\) em \(C\);
\end{enumerate}

    \begin{enumerate}
\def\labelenumi{\arabic{enumi})}
\setcounter{enumi}{1}
\item
  \(A\times B=\{(a,a), (a,b)\}\) é uma \textbf{relação} com origem em
  \(A\) e destino em \(B\);
\end{enumerate}

    \begin{enumerate}
\def\labelenumi{\arabic{enumi})}
\setcounter{enumi}{2}
\item
  Considerando o domínio \(A\) e o contradomínio \(B\), a
  \textbf{relação} de \emph{igualdade} é \(\{(a,a)\}\);
\end{enumerate}

    \begin{enumerate}
\def\labelenumi{\arabic{enumi})}
\setcounter{enumi}{3}
\item
  \(\{(0,1),(0,2),(1,2)\}\) é a \textbf{relação} ``\emph{menor que}''
  (\(<\)) de \(C\) em \(C\);
\end{enumerate}

    \begin{enumerate}
\def\labelenumi{\arabic{enumi})}
\setcounter{enumi}{4}
\item
  \(\{(0,a),(1,b)\}\) é uma \textbf{relação} de \(C\) em \(B\).
\end{enumerate}

\subsection{Domínio de definição, conjunto
imagem}

\subsubsection{Definição}

Seja \(R: A\rightarrow B\) uma \textbf{relação}. Então:

\begin{enumerate}
\def\labelenumi{\arabic{enumi})}
\item
  Se \((a,b)\in R\), então afirma-se que \(R\) está \emph{definida} para
  \(A\), e que \(b\) é a imagem de \(a\);
\end{enumerate}

    \begin{enumerate}
\def\labelenumi{\arabic{enumi})}
\setcounter{enumi}{1}
\item
  O conjunto de todos os elementos de \(A\) para o qual \(R\) está
  definida é denomindo \emph{domínio de definição};
\end{enumerate}

    \begin{enumerate}
\def\labelenumi{\arabic{enumi})}
\setcounter{enumi}{2}
\item
  O conjunto de todos os elementos de \(B\), imagem de \(R\), é
  denomindo \emph{conjunto imagem}.
\end{enumerate}

    \textbf{Exemplo}

Sejam \(A=\{a\}\), \(B=\{a,b\}\) e \(C=\{0,1,2\}\), então:

    \begin{enumerate}
\def\labelenumi{\arabic{enumi})}
\item
  Para a \textbf{relação} \(\emptyset\): \(A\rightarrow B\), o domínio
  de definição e o conjunto imagem são \emph{vazios};
\end{enumerate}

    \begin{enumerate}
\def\labelenumi{\arabic{enumi})}
\setcounter{enumi}{1}
\item
  Para a endorrelação\(^*\) (C,\textless), definida por
  \(\{(0,1),(0,2),(1,2)\}\), o domínio de definição é \(\{0,1\}\) e a
  imagem é \(\{1,2\}\); --- \(^*\) Endorrelação ou autorrelação é uma
  relação \(R:A\rightarrow A\) (origem e destino no mesmo conjunto).
\end{enumerate}

    \begin{enumerate}
\def\labelenumi{\arabic{enumi})}
\setcounter{enumi}{2}
\item
  Para a relação \(=: A\rightarrow B\), o conjunto \{A\} é o domínio de
  definição e o cojunto imagem.
\end{enumerate}

    \textbf{Exemplo - Banco de Dados Relacional}*

Sejam o conjunto de países
\[P=\{\text{"Brasil"},\text{"Turquia"},\text{"Alemanha"}, \text{"Coreia do Sul"},\text{"Austrália"}\}\]
e o conjunto de continentes
\[C=\{\text{"América"},\text{"Oceania"}, \text{"África"},\text{"Ásia"}, \text{"Europa"}\}\].

A \textbf{relação}
\(R=\{(x,y)\in P\times C\ |\ x\text{ localiza-se em }y\}\) resultaria

\[R=\{(\text{"Brasil"},\text{"América"}),(\text{"Turquia"},\text{"Europa"}),\]
\[(\text{"Turquia"},\text{"Ásia"}),(\text{"Alemanha"},\text{"Europa"}),\]
\[(\text{"Coreia do Sul"},\text{"Ásia"}),(\text{"Austrália"},\text{"Oceania"})\}\]

    \[R=\{(\text{"Brasil"},\text{"América"}),(\text{"Turquia"},\text{"Europa"}),\]
\[(\text{"Turquia"},\text{"Ásia"}),(\text{"Alemanha"},\text{"Europa"}),\]
\[(\text{"Coreia do Sul"},\text{"Ásia"}),(\text{"Austrália"},\text{"Oceania"})\}\]

\begin{longtable}[]{@{}lc@{}}
País & Continente\tabularnewline
\endhead
``Brasil'' & ``América''\tabularnewline
``Turquia'' & ``Europa''\tabularnewline
``Turquia'' & ``Ásia''\tabularnewline
``Alemanha'' & ``Europa''\tabularnewline
``Coreia do Sul'' & ``Ásia''\tabularnewline
``Austrália'' & ``Oceania''\tabularnewline
\end{longtable}

    \hypertarget{relauxe7uxe3o-inversa}{%
\subsection{Relação Inversa}\label{relauxe7uxe3o-inversa}}

Seja \(R\) uma relação. A \textbf{inversa} de \(R\), denotada por
\(R^{-1}\), é a \textbf{relação} formada invertendo-se todos os pares
ordenados em \(R\), e sua fórmula é

\[R^{-1}=\{(x,y)\ :\ (y,x)\in R\}\]

    \textbf{Exemplo}

Seja

\[R=\{(1,5),(2,6),(3,7),(3,8)\}\]

então,

\[R^{-1}=\{(5,1),(6,2),(7,3),(8,3)\}\]

\subsection{Exercícios}

\begin{enumerate}
\def\labelenumi{\arabic{enumi}.}
\item
  Sejam \(A=\{2,3,4,5\}\) e \(B=\{3,4,5,6,10\}\), para cada uma das
  seguintes \textbf{relações}:
\end{enumerate}

\begin{enumerate}
\item
  \(R_1=\{(x,y)\in A\times B \ :\ x|y\}\)
\item
  \(R_2=\{(x,y)\in A\times B \ :\ x=y+1\}\)
\item
  \(R_3=\{(x,y)\in A\times B \ :\ xy=12\}\)
\item
  \(R_4=\{(x,y)\in A\times B \ :\ x\geq y\}\)
\end{enumerate}

\begin{itemize}
\item
  explicite as tuplas da \textbf{relação};
\item
  faça um representação gráfica no plano cartesiano;
\item
  determine o domínio de definição;
\item
  determine o conjunto imagem.
\end{itemize}

\begin{enumerate}
\def\labelenumi{\arabic{enumi}.}
\setcounter{enumi}{1}
\item
  {[}Scheinerman{]} Determine \(R^{-1}\) para cada uma das seguintes
  relações:
\end{enumerate}

\begin{enumerate}
\def\labelenumi{\alph{enumi})}
\item
  \(R_1=\{(1,2),(2,3),(3,4)\}\)
\item
  \(R_2=\{(1,1),(2,2),(3,3)\}\)
\item
  \(R_3=\{(x,y)\ :\ x,y \in \mathbb{Z}, x-y=1\}\)
\item
  \(R_4=\{(x,y)\ :\ x,y \in \mathbb{N}, x|y\}\)
\item
  \(R_5=\{(x,y)\ :\ x,y \in \mathbb{Z}, xy>0\}\)
\end{enumerate}

\begin{center}\rule{0.5\linewidth}{0.5pt}\end{center}

\begin{itemize}
\item
  \(x|y\) significa que \(x\) é divisível por \(y\).
\end{itemize}

    \hypertarget{bibliografia}{%
\subsection{Bibliografia}\label{bibliografia}}

\begin{itemize}
\item
  Edward R. Scheinerman.
  \href{https://www.amazon.com.br/Matem\%C3\%A1tica-discreta-introdu\%C3\%A7\%C3\%A3o-Edward-Scheinerman/dp/8522125341/}{``Matemática
  discreta: Uma introdução''}, 2a. edição. Cengage Learning, 2011.
\item
  Paulo Blauth Menezes.
  \href{https://www.amazon.com.br/Matem\%C3\%A1tica-Discreta-para-Computa\%C3\%A7\%C3\%A3o-Inform\%C3\%A1tica/dp/8582600240}{``Matematica
  Discreta para Computação e Informática''}, 3a. edição. Editora
  Bookman, 2010.
\end{itemize}
