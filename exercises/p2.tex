\documentclass[10pt,a4paper]{article}

\usepackage[brazil]{babel}
\usepackage{fontspec}
\usepackage{enumerate}

\def\tupla#1{\langle #1\rangle}
\def\mod#1#2{$#1\Delta#2$}

\newcounter{exn}\setcounter{exn}{0}

\def\ex#1{\paragraph{Questão \arabic{exn}\addtocounter{exn}{1}.} \marginpar{\footnotesize \hfill #1 ponto(s)}}


\title{$2^a$ Prova de Matemática Discreta}
\author{prof. Adriano J. Holanda}
\date{29 de novembro de 2016}


\begin{document}
\maketitle

\noindent Faculdade ``Dr. Francisco Maeda'' -- FAFRAM\\
\noindent Curso: Sistemas de Informação, 4$^o$ Ciclo\\
\begin{minipage}{.225\textwidth}
\noindent Nome completo:
\end{minipage}
\begin{minipage}{.8\textwidth}
  {$\rightarrow$}~\hrule
\end{minipage}

\ex{2} Desenvolva as seguintes expressões utilizando o triângulo de Pascal:

\begin{enumerate}[(a)]
\begin{minipage}{.5\textwidth}
\item $\left( \frac{x^3}{2} - \frac{1}{y^2}\right)^4$ 
\end{minipage}
\begin{minipage}{.5\textwidth}
\item $\left( 2x^2 + \frac{1}{x^2}\right)^3$ 
\end{minipage}
\end{enumerate}

\ex{3} Ache o máximo divisor comum (mdc), demonstrando os passos
intermediários das operações, utilizando o algoritmo de Euclides para
os seguintes valores:

\begin{enumerate}[(a)]
\begin{minipage}{.5\textwidth}
\item mdc(10, 4)
\item mdc(17, 11)
\item mdc(121, 15)
\end{minipage}
\begin{minipage}{.5\textwidth}
\item mdc(21, 11)
\item mdc(21, 7)
\item mdc(31, 26)
\end{minipage}
\end{enumerate}

\ex{3} Efetue as operações {\tt módulo} ($\Delta$) a seguir, mostrando
os cálculos intermediários:

\begin{enumerate}[(a)]
\begin{minipage}{.5\textwidth}
\item \mod{11}{3}
\item \mod{11}{-3}
\item \mod{-11}{3}
\end{minipage}
\begin{minipage}{.5\textwidth}
\item \mod{-11}{-3}
\item \mod{555}{333}
\item \mod{1000}{321}
\end{minipage}
\end{enumerate}

\ex{2} No mapeamento do endereço da memória principal para a memória
cache podemos utilizar a operação módulo para determinar as posições
de escrita e busca. Supondo que em nosso sistema de computação, a
memória principal tenha {\bf 8} endereços e a memória cache {\bf 4},
faça o esquema de distribuição da memória principal para a cache
utilizando a configuração associativa de {\bf 1} via, também chamado 
mapeamento direto.

\vfill

\begin{flushright}
\bf\large  Boa Prova
\end{flushright}

\end{document}

