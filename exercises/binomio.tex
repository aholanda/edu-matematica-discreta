\def\exercise{\paragraph{Exercício. }}
%\def\correta{$\leftarrow$}
\def\correta{}

\section*{Coeficientes binomiais}

%{Fonte:~\cite{santos2007}}

\exercise{} Desenvolver as seguintes pot\^encias utilizando o
tri\^angulo de Pascal:

\begin{enumerate}[(a)]
\item $\left( \frac{x^3}{2} + 1\right)^5$
\item $\left( 2y + 3x\right)^4$
\item $\left( 2a - {3 \over b}\right)^3$
\item $\left( \frac{1}{y} - y \right)^6 $
\end{enumerate}

\exercise{} Determinar os seguintes coeficientes binomiais:

\begin{enumerate}
\item $0 \choose 0$
\item $ n \choose n-1 $
\item $ 10 \choose 8$
\item $ 100 \choose 10$
\item $ 1000 \choose 2$
\end{enumerate}


\exercise{} Quantos números de 3 algarismos podemos formar com os
dígitos 2, 3, 4, 5, 8, 9, supondo que não haja repetição de dígitos
nos números formados.

\exercise{} Numa sala há 7 homens e 5 mulheres. Quantas comissões de 5
pessoas podem ser formadas:

\begin{enumerate}[(a)]
\item sem restrições?
\item se da comissão fizer parte 3 homens e 2 mulheres?
\end{enumerate}

\exercise{} Supondo que as placas dos veículos contêm 3 letras (dentre
as 26 disponíveis), seguidas de 4 dígitos numéricos, quantas são as
combinações possíveis, supondo que não haja repetição de letras e
números nas placas?


\exercise{} {\bf Bônus -- não será aplicado na prova, mas serve como
  verificação de domínio da matéria}. Use o triângulo de Pascal para
explicar o fato de que $11^4 = 14641$.


\exercise{} [3,0 pontos] Assinale a alternativa correta para a
expansão das potências abaixo (Dica: Usar o triângulo de
Pascal):

 \hfil$\left(2y + x\right)^3$\hfill

 \begin{enumerate}[(a)]
 \item $2y^3 + 12y^2x + 6x^2y + x^3$
 \item $2y^3 + 6y^2x + x^2y + x^3$
 \item $2y^3 + 6y^2x + 6x^2y + x^3$ \correta
 \item $8y^3 + 12y^2x + 6yx^2 + x^3$
 \item $8y^3 + 6y^2x + 12x^2y + x^3$
 \end{enumerate}

 \hfil$\left(x + y\right)^5$\hfill

 \begin{enumerate}[(a)]
 \item $x^5 + 5x^4y + 10x^3y^2 + 5x^2y^3 + 5xy^4 + y^5$
 \item $x^5 + 5x^4y + 10x^3y^2 + 10x^2y^3 + 5xy^4 + y^5$ \correta
 \item $x^5 + 5x^4y + 5x^3y^2 + 10x^2y^3 + 5xy^4 + y^5$
 \item $x^5 + x^4y + 5x^3y^2 + 5x^2y^3 + xy^4 + y^5$
 \item $x^5 + x^4y + 10x^3y^2 + 10x^2y^3 + xy^4 + y^5$
 \end{enumerate}

 \exercise{}  Assinale a alternativa correta para os
 coeficientes binomiais a seguir:

$$ {30 \choose 2}$$
\begin{enumerate}[(a)]
\item 233
\item 481
\item 435 \correta
\item 321
\item 400
\end{enumerate}

\newpage
$${21 \choose 3}$$

\begin{enumerate}[(a)]
\item 1331
\item 1330 \correta
\item 1300
\item 1322
\item 1301
\end{enumerate}

\exercise{}  quer usar as letras de seu nome embaralhadas como
senha de usuário em um sistema de autenticação, podendo as letras
serem maiúsculas ou minúsculas. Qual o número de combinações possível
para a senha de Pedro?

\begin{enumerate}
\item 252 \correta
\item 250
\item 232
\item 244
\item 288
\end{enumerate}


\exercise{}  decide aperfeiçoar seu esquema de senhas adicionando
3 números no final da senha. Qual o número de combinações com a adição
destes números?

\begin{enumerate}[(a)]
\item 32.260
\item 24.000
\item 30.240 \correta
\item 16.600
\item 32.000
\end{enumerate}
