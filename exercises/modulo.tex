\newcounter{exno}\setcounter{exno}{0}
\def\exerc{\paragraph{Exercício \arabic{exno}.}
  \addtocounter{exno}{1}}


\section*{Operação Módulo}

\def\modsym{\Delta} \def\mod#1#2{$#1\modsym#2$} \exerc{} Realize as
seguintes operações módulo ($\modsym$), mostrando os cálculos 
intermediários:

\begin{enumerate}[(a)]
\begin{minipage}{.5\textwidth}
\item \mod{12}{5}
\item \mod{12}{-5}
\item \mod{-12}{5}
\item \mod{-12}{-5}
\item \mod{5}{12}
\item \mod{4}{12}
\item \mod{6}{12}
\item \mod{11}{12}
\end{minipage}
\begin{minipage}{.5\textwidth}
\item \mod{33}{3}
\item \mod{14}{1}
\item \mod{4}{3}
\item \mod{-39}{21}
\item \mod{11}{4}
\item \mod{555}{333}
\item \mod{333}{555}
\item \mod{1000}{291}
\end{minipage}
\end{enumerate}

\exerc{} No mapeamento do endereço da memória principal para a memória
cache podemos utilizar a operação módulo para determinar as posições
de escrita e busca. Supondo que em nosso sistema de computação, a
memória principal tenha {\bf 16} endereços e a memória cache {\bf 8},
faça o esquema de distribuição da memória principal para a cache
utilizando as configurações associativa de {\bf 4} e {\bf 8} vias.


