\documentclass[10pt,a4paper]{article}

\usepackage{fontspec}
\usepackage[brazil]{babel}
\usepackage{enumerate}
\usepackage{a4wide}

\def\tupla#1{\langle #1\rangle}

\newcounter{exn}\setcounter{exn}{0}

\def\ex#1{\paragraph{Questão \arabic{exn}\addtocounter{exn}{1}.} \marginpar{\footnotesize \hfill #1 ponto(s)}}

\title{$1^a$ Prova de Matemática Discreta}
\author{prof. Adriano J. Holanda}
%\date{28 de novembro de 2012}
\date{27 de setembro de 2016}


\begin{document}
\maketitle

\noindent Faculdade ``Dr. Francisco Maeda'' -- FAFRAM\\
\noindent Curso: Sistemas de Informação, Ciclo 4\\
\begin{minipage}{.225\textwidth}
\noindent Nome completo:
\end{minipage}
\begin{minipage}{.8\textwidth}
  {$\rightarrow$}~\hrule
\end{minipage}

\ex{2} Dados os conjuntos $A=\{1\}$; $B=\{2\}$ e $C = \{\{1\}, 1\}$, classifique
como verdadeira (V) ou falsa (F) as seguintes afirmações:

\bigskip

\begin{minipage}{0.45\textwidth}
\begin{enumerate}[a. (\ )]
\item $A\subset B$ %X
\item $A\subseteq B$%X
\item $A = B$
\item $A \in B$
\item $A\subset C$%X
\item $A\subseteq C$%X
\item $A\in C$%X
\end{enumerate}
\end{minipage}
\begin{minipage}{0.45\textwidth}
\begin{enumerate}[A. (\ )]
\item $A = C$
\item $1 \in A$%X
\item $1 \in C$%X
\item $\{1\} \in A$
\item $\{1\} \in C$%X
\item $\emptyset \notin C$%X
\item $\emptyset \subseteq C$%X
\end{enumerate}
\end{minipage}


\ex{4} Sejam $A=\{2,3,4,5,12\}$ e $B=\{3,4,6,10\}$. Para cada uma
das seguintes relações:
\label{ex:rel}

\begin{enumerate}[(a)]
\item $R_1=\{\langle x,y\rangle\in A\times B : x \mbox{ é divisível
por } y\}$
\item $R_2=\{\langle x,y\rangle\in A\times B : x \mbox{ é ímpar } \land
y \mbox{ é par}\}$
%\item $R_3=\{\langle x,y\rangle\in A\times B : x \mbox{ e } y \mbox{
%sejam números primos}\}$
%\item $R_4=\{\langle x,y\rangle\in A\times B : y=2*x\}$
%\item $R_5=\{\langle x,y\rangle\in A\times B : x=y \lor
%  y=x+1\}$~\footnote{$\lor$ - conectivo lógico {\tt ou}.}
\end{enumerate}

\begin{enumerate}[(1)]
\item explicite os pares ordenados da relação;
\item faça a representação gráfica;
\item determine o domínio da definição;
\item determine o conjunto imagem.
\end{enumerate}

\ex{4} Determinar se as seguintes relações são reflexiva, simétrica e
transitiva e de equivalência:

\begin{enumerate}[(a)]
\item
$R_1=\{\tupla{1,3},\tupla{1,1},\tupla{2,2},\tupla{3,1},\tupla{3,3}\}$
\item
$R_2=\{\tupla{1,1},\tupla{2,2},\tupla{3,3},\tupla{1,2},\tupla{2,3}\}$
\item $R_3=\{\tupla{1,1},\tupla{1,2},\tupla{2,3},\tupla{3,1}\}$
\item $R_4 = A\times A$
%\item $R_5 = \emptyset$
\end{enumerate}

\noindent Explique os critérios utilizados para a classificação das relações.

\vfill

\begin{flushright}
\bf\large  Boa Prova
\end{flushright}

\end{document}

\ex{1,5} Dados os conjuntos $A=\{1,2,3\}$, $B=\{3,4,5\}$ e
$C=\{4,5,7\}$, realize as seguintes operações:

\begin{enumerate}[(a)]
\item $A\cup B $
\item $A\cap B $
\item $A\cap C $
\item $A\cap B\cap C $
\item $A - B $
\item $B - A $
\item $A\Delta B $
\item $A\cup B \cap C$
\item $A\cap B \cup C$
%\item $ B \Delta C$
\end{enumerate}

\ex{2,5} Dados os conjuntos $A=\{2,3,6,7,8,11\}$,
$B=\{0,1,4,5,8,9,12,15\}$, obtenha as seguintes 
relações:

\def\tuple#1{\langle{#1}\rangle}
\begin{enumerate}[(a)]
\item $R_1=\{\tuple{x,y}\in A\times B:x=\sqrt{y}\}$
\item $R_2=\{\tuple{x,y}\in A\times B:x=y^2\}$
\item $R_3=\{\tuple{x,y}\in A\times B:x=y+1\}$
 \item $R_4=\{\tuple{x,y}\in A\times B:x \mbox{ é par e primo } \land
  y\mbox{ é impar e primo }\}$
\item $R_5=\{\tuple{x,y}\in A\times B:x=y-4\}$
%\item $R_6=\{\tuple{x,y}\in A\times B:x \mbox{ é divisível por }
 % y\}$
%\item $R_7=\{\tuple{x,y}\in A\times B:y=x*2\}$
%\item $R_8=\{\tuple{x,y}\in A\times B:x\mbox{ é par } \land y \mbox{ é
%    negativo }\}$
%\item $R_9=\{\tuple{x,y}\in A\times B:x$
%\item $R_{10}=\{\tuple{x,y}\in A\times B:x=y^3\}$
\end{enumerate}
