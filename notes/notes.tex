
%%% Local Variables: 
%%% mode: latex
%%% TeX-master: t
%%% End: 

\documentclass[twoside,a4wide,12pt]{book}

\usepackage[portuguese]{babel}
\usepackage[utf8]{inputenc}
\usepackage{enumerate}
\usepackage{amsthm}
\usepackage{amssymb}

\newtheorem{definition}{Definição}

%def exercise
\newcounter{exc}\setcounter{exc}{0}
\def\exercise{\bigskip
\addtocounter{exc}{1}\paragraph{Exercício \arabic{exc}.}}

%defpar
\newcommand\ordpar[1]{{\langle #1\rangle}}%

%def newMarginpar
\setlength{\marginparwidth}{1.2in} \let\oldmarginpar\marginpar
\renewcommand\marginpar[1]{\-\oldmarginpar[\raggedleft\footnotesize
  #1]%
  {\raggedright\scriptsize #1}}

\author{Adriano J. Holanda}
\date{\today}
\title{Matemática Discreta}

\begin{document}

\chapter{Noções básicas de teoria dos conjuntos}

\section{Introdução}

\begin{definition}
  \label{def:set}
  Um {\bf conjunto\/} é uma coleção de zero ou mais objetos distintos,
  chamados {\bf elementos\/} do conjunto, os quais não possuem
  qualquer ordem associada.
\end{definition}


\noindent Exemplos:
\begin{tabbing}
\hspace{1cm}\={\bf Vogais}$=\{a,e,i,o,u\}$\\
\>{\bf Dígitos}$=\{0,1,2,3,4,\ldots,9\}$\\
\>{\bf Pares}$=\{0,2,4,6,\ldots\}$\\
\>{\bf DiasSemana}$=\{seg,ter,qua,qui,sex,sab,dom\}$\\
\>{\bf Meses}$=\{jan,fev,mar,abr,mai,jun,jul,ago,set,out,nov,dez\}$
\end{tabbing}

A Definição~\ref{def:set} é uma tentativa de definir intuitivamente,
já que a frase ``conjunto é uma coleção'' não explica muito. O
conceito de conjunto é tão fundamental que não tentaremos defini-lo
rigorosamente.


\paragraph{Representação.} Um conjunto pode ser representado de duas
maneiras:

\begin{description}
\item[1. Enumeração:] $\{$Bruna, João, Pedro, Paulo, Maria, Joana, Júlia$\}$
\item[2. Descrição:] $\{x|x$ cursa a disciplina de Matemática Discreta
  da Instituição$\}$
\end{description}

Na enumeração fazemos uma listagem de todos os elementos do
conjunto. Se a lista for muito grande as reticências podem ser usadas
indicando que o conjunto prossegue segundo alguma ``lei de geração'',
como por exemplo o conjunto dos números pares:
\begin{center}
  $\{0,2,4,6,8,10,\ldots\}$.
\end{center}

A representação por descrição utiliza alguma propriedade comum a todos
os elementos do conjunto para selecionar seus elementos. No exemplo a
propriedade é ``todos os alunos que cursam a disciplina Matemática
Discreta''. No exemplo que usa enumeração estes alunos foram listados
individualmente.

\section{Pertinência}

Deixaremos também indefinida a noção de {\bf pertinência} $\in$, onde
\begin{center}
  $x \in C$
\end{center}
significa que $x$ pertence ao conjunto $C$. Caso contrário, afirmamos
que $x$ {\it não\/} pertence ao conjunto $C$, denotado por
\begin{center}
  $x\notin C$.
\end{center}

\noindent Exemplos:
\begin{tabbing}
  \hspace{1cm}\=1. $a\in$ {\bf Vogais}\\
  \>2. $a\notin$ {\bf Dígitos}\\
\end{tabbing}

\subsection{Conjuntos especiais}

\paragraph{Conjunto vazio.} O conjunto sem elementos $\{\ \}$ é
representado pelo símbolo~$\emptyset$.

\bigskip
\noindent Exemplos:
\begin{tabbing}
  \hspace{1cm}\=1. $\{x|x$ é par e ímpar$\}$\\
  \>2. $\{$conjuntos de todas as pessoas com mais de 5 metros de altura$\}$\\
\end{tabbing}

\section{Conjuntos finitos e infinitos}

Um conjunto é finito quando seus elementos podem ser listados
exaustivamente, caso contrário, é denotado como infinito.

\noindent Exemplos
\noindent Conjunto finito
\begin{tabbing}
\hspace{1cm}\=1. {\bf Vogais}$=\{a,e,i,o,u\}$\\
\>2. {\bf Dígitos}$=\{0,1,2,3,4,\ldots,9\}$\\
\>3. {\bf C}$=\{x|x$ é corintiano$\}$\\
\end{tabbing}

\noindent Conjunto infinito
\begin{tabbing}
\hspace{1cm}\=1. $\mathbb{N}$ \hspace{6cm}\= conjunto dos números naturais\\
\>2. $\mathbb{Z}$\>conjunto dos números inteiros\\
\>3. $\mathbb{Q}$\>conjunto dos números racionais\\
\>4. $\mathbb{I}$\>conjunto dos números irracionais\\
\>5. $\mathbb{R}$\>conjunto dos números reais\\
\>6. {\bf Pares}$=\{y|y=2x$ e $x \in \mathbb{N}\}$\\
\end{tabbing}

\section{Relação entre conjuntos ou continência}

Se cada elemento de um conjunto $A$ for também elemento de outro
conjunto $B$ dizemos que {\it está contido} em $B$, ou que $A$
{\it é um subconjunto} de $B$, e denotamos este fato através da
notação:
$$A \subseteq B.$$

\noindent ou alternativamente, denotamos que $B$ contém $A$ por:
$$B\supseteq A.$$

\noindent Exemplos:

\begin{tabbing}
\hspace{1cm}\=1. $\{a,b\}\subseteq\{b,a\}$\\
\>2. $\{a,b\}\subseteq\{a,b,c\},\{a,b\}\subset \{a,b,c\}$\\
\>3. $\{1,2,3\}\subseteq\mathbb{N},\{1,2,3\}\subset\mathbb{N}$\\
\>4. $\mathbb{N}\subseteq\mathbb{Z}, \mathbb{N}\subset\mathbb{Z}$\\
\>5. $\emptyset\subseteq\{a,b,c\},\emptyset\subset\{a,b,c\}$\\
\>6. $\emptyset\subseteq\mathbb{N},\emptyset\subset\mathbb{N}$
\end{tabbing}


A inclusão própria ocorre quando todos os elementos de $A$ são
elementos de $B$, porém nem todos elementos de $B$ são elementos de
$A$, denotamos este fato da seguinte maneira:
$$A \subset B.$$

Dizemos que $A$ é um subconjunto próprio de $B$, ou que $A$ está
propriamente contido em $B$ e equivale a notação: $$A\subseteq B\ e\
A\neq B.$$

Se $A$ e $B$ possuem exatamente os mesmos elementos, eles são o mesmo
conjunto e representamos da seguinte forma:
$$A=B.$$

\noindent Exemplos:

\begin{tabbing}
\hspace{1cm}\=1. $\{1,2,3\}=\{x\in\mathbb{N}\mid x>0\ e\ x<4b\}$\\
\>2. $\mathbb{N}=\{x\in\mathbb{Z}\mid x\geq 0\}$\\
\>3. $\{1,2,3\} = \{3,3,2,2,2,1,1,1,1,1\}$\\
\end{tabbing}

\exercise{} Classifique as seguintes afirmações em verdadeiras ou
falsas:

\begin{enumerate}[(a)( )]
\item $c\in\{a,c,e\}$
\item $e\notin\{a,b,c\}$
\item $\{0,1,2\}\subset\{0,1,2\}$
\item $\{0,1,2\}\subseteq\{0,1,2\}$
\item $\{a,b\}\subseteq\{a,b,c\}$
\item $a\in\{b,\{a\}\}$
\item $\{a\}\in\{b,\{a\}\}$
\item ${a}\in\{c,\{b\},a\}$
\item $\emptyset\in\{a,b,c\}$
\item $\{0,1,2\}\subseteq\{3,2,5,4,6\}$
\end{enumerate}

\section{Operações sobre conjuntos}
Outra maneira de caracterizar conjuntos, além da enumeração e
descrição, é gerá-los através de operações. Se quisermos um conjunto
formado pelos elementos de $A$ e $B$, aplicamos a operação de {\it
  união}\marginpar{união} sobre os conjuntos denotando por $$A \cup B.$$

Por definição, o conjunto $A \cup B$ contém todos os elementos que são
elementos de $A$ ou de $B$.

$$A\cup B = \{x\mid x\in A\ ou\ x\in B\}$$

\noindent Exemplo:
\begin{tabbing}
\hspace{1cm}\=Se $A=\{1,2,3\}$ e $B=\{a,b,c\}$, então\\
\>$A \cup B = \{a,b,c,1,2,3\}$\\
\end{tabbing}

Outra operação a ser definida é a
\emph{intersecção}\marginpar{intersecção}, denotada por
$$A\cap B$$
que denota os elementos que pertencem tanto a $A$ quanto a $B$, ou
seja,
$$A\cap B=\{x\mid x\in A\ e\ x \in B\}.$$


\noindent Exemplo:
\begin{tabbing}
\hspace{1cm}\=Se $A=\{1,2,3\}$ e $B=\{0,1,2,4,6\}$, então\\
\>$A \cup B = \{1,2\}$\\
\end{tabbing}

Dado um universo $U$ e um conjunto $A$ contido em $U$, operação de
\emph{complemento}\marginpar{complemento} de $A$, denotada por
$\overline{A}$ é o conjunto de todos os elementos que não pertencem a
$A$ e pode ser definida como:
$$\overline{A}=\{x|x\in U\ e\ x\notin A\}$$

\noindent Exemplo:
\begin{tabbing}
\hspace{1cm}\=$\overline{U}=\emptyset$\\
\>$\overline{\emptyset} = U$\\
\>Supondo o conjunto universo como \={\bf
  Digitos=}$\{0,1,2,3,4,5,6,7,8,9\}$ e\\
\>\>$A=\{0,1,3\}$, então\\
\>\>$\overline{A}=\{2,4,5,6,7,8,9\}$
\end{tabbing}

Falta definir a operação da \emph{diferença}\marginpar{diferença} como
$$A-B=\{x\mid x\in A\ e\ x\notin B\},$$
ou seja, dados os conjuntos $A$ e $B$, o elemento $x$ pertence ao
conjunto $A-B$ se pertencer ao conjunto $A$ e não pertencer ao
conjunto $B$.\\

\noindent Exemplo:
\begin{tabbing}
\hspace{1cm}\=Se $A=\{0,2,4,6\}$ e $B=\{1,2,3,4\}$, então\\
\>$A - B = \{0,6\}$\\
\end{tabbing}

\subsection{Conjunto das partes}

Dado um conjunto $A$, podemos também formar o \emph{conjunto potência}
\marginpar{potência}
de $A$ (ou conjunto das partes de $A$), que corresponde ao conjunto de
todos os subconjuntos de $A$, e denotaremos por
$\mathcal{P}(A)$. Podemos defini-lo assim: 

$$\mathcal{P}(A)=\{\mathcal{X}\mid\mathcal{X}\in A\}.$$

\noindent Por exemplo, se $A=\{1,2\}$, temos que
$$\mathcal{P}(A)=\{\emptyset,\{1\},\{2\},\{1,2\}\}.$$

De um modo geral, se um conjunto $A$ tem $n$ elementos,
$\mathcal{P}(A)$ terá $2^n$ elementos.

\subsection{Produto cartesiano}

Iremos chamar um conjunto de dois elementos de \emph{par}, e como
qualquer conjunto não possuem relação de ordem, o seguinte fato é
válido:
$$\{a,b\}=\{b,a\}.$$

Por outro, se quisermos que os elementos de um par tenham ordem,
deveremos introduzir a noção de \emph{par ordenado}\marginpar{par
  ordenado}, que é um tipo especial de conjunto. Assim o par ordenado
$\langle x,y\rangle$ é definido da seguinte maneira:
$$\langle x,y\rangle=\{\{x\},\{x,y\}\}.$$

A partir desta definição notamos com mais clareza que $\langle
a,b\rangle$ é diferente de $\langle a,b\rangle$, pois
$$\langle a,b\rangle=\{\{a\},\{a,b\}\}$$
\noindent é diferente de
$$\langle b,a\rangle=\{\{b\},\{b,a\}\}.$$

A mesma definição pode ser estendida para um conjunto de três
elementos ordenados, chamado tripla, por exemplo o conjunto $\langle
a,b,c\rangle$ é diferente de $$\langle c,a,b\rangle$$. De modo
análogo, temos as quádruplas ordenadas e, no caso geral, sequências
ordenadas $$\langle a_1,\ldots,a_n\rangle$$ de $n$ elementos -- as
\emph{n-uplas}, ou \emph{ênuplas}.\marginpar{n-uplas}.

Deste modo, o produto cartesiano de $A$ e $B$, denotado por $A\times
B$, é o conjunto de pares ordenados $\langle x,y\rangle$, tal que
$x$ pertence a $A$  e y pertence a $B$, ou seja:

$$A\times B=\{\langle a,b\rangle\mid x\in A\ e\ y\in B\}.$$

\noindent Exemplo:
\begin{tabbing}
\hspace{1cm}\=Se $A=\{1,2\}$ e $B=\{a,b\}$, então\\
\>$A\times B = \{\ordpar{1,a}, \ordpar{1,b},\ordpar{2,a},\ordpar{2,b}\}$\\
\end{tabbing}

Generalizando para as triplas temos que o produto cartesiano $A\times
B\times C$ é composto pelas triplas ordenadas, tendo o primeiro
elemento extraído do conjunto $A$, o segundo de $B$ e o terceiro de
$C$.

Podemos fazer o produto cartesiano de um conjunto por ele mesmo,
denotado por $A\times A=A^2$; $A\times A \times A=A^3$; $\ldots$; $A
\times A$ n vezes é igual a $A^n$, e $A^1$ é ele próprio.

\begin{thebibliography}{99}

\bibitem[Menezes, 2010]{paulo2010} Paulo Blauth Menezes. \emph{Matemática discreta para
    computação e matemática}. Editora Bookman, $3^a$ edição, 2010.

\bibitem[Mortari, 2001]{mortari2001} Cezar A. Mortari. \emph{Introdução à lógica}. Editora UNESP,
  2001.

\end{thebibliography}

\end{document}
