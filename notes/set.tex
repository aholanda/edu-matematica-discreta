\section{Noções de Teoria dos Conjuntos}

\paragraph{Definição.} 
\textbf{Teoria dos conjuntos}\footnote{A teoria de 
conjuntos em questão refere-se
ao sistema ZFC, ou seja, o conjunto 
de axiomas de Zermelo-Fraenkel
somados ao axioma da escolha.}
é o ramo da matemática que estuda os conjuntos, 
que são coleções de objetos.

\hypertarget{caracteruxedsticas}{%
\subsection{Características}\label{caracteruxedsticas}}

\begin{itemize}
\item
  É o sistema fundamental empregado na matemática
  atualmente;
\item
  A linguagem da teoria de conjuntos pode ser usada na definição de
  quase todos os objetos matemáticos.
\end{itemize}

\subsection{Exemplos de Conjuntos}

\begin{longtable}[]{@{}ll@{}}\hline
\hfil Notação & \hfil Descrição\\\hline
\(\{\)``aba'' \(,\) "carro\("\}\) & conjunto formado pelas
\emph{strings} ``aba'' e ``carro''\tabularnewline
\(\{1,2,3\}\) & conjunto formado pelos números inteiros
\(1, 2, 3\)\tabularnewline
	\(\mathbb{N}\) & conjunto dos números \textbf{naturais}
\(0,1,2,\dots\)\tabularnewline
	\(\mathbb{R}\) & conjunto dos números \textbf{reais}\tabularnewline
\(\emptyset\) & conjunto \textbf{vazio}, possui nenhum
elemento\tabularnewline
\(\{x:x\) é par\(\}\) & conjunto dos números
\textbf{pares}\tabularnewline
\(\{\{1\},2\}\) & conjunto contendo número \(2\) e o conjunto com o
número \(1\)\\\hline
\end{longtable}

\subsection{Conceitos Básicos}

\subsubsection{Tamanho de um Conjunto}

Dado um conjunto \(A\), tamanho do conjunto \(A\) é representado pela
notação

\[|A|.\]

Examplo:

\(A=\{2,4,5,7,8,11,13,15\}\) \(|A|=8\)

\subsubsection{Relação de
Pertinência}

\(o \in A\) - o objeto \(o\) \textbf{pertence} ao conjunto \(A\).

Exemplo:

O elemento \(1\) \textbf{pertence} ao conjunto \(\{1,2,3\}\), mas \(4\)
\textbf{não}.


\subsubsection{Relação de Inclusão}

\(A\subseteq B\) - o conjunto \(A\) \textbf{está contido} no conjunto
\(B\), ou seja, \(A\) é formado por um \textbf{subconjunto} de \(B\).

Exemplo:

\(\{1,2\}\) é um \textbf{subconjunto} de \(\{1,2,3\}\), mas \(\{3,4\}\)
\textbf{não}.

\subsubsection{Subconjunto}

Sejam os conjuntos \(A\) e \(B\). Dizemos que \(A\) é um
\textbf{subconjunto} de \(B\), se e somente se, \textbf{todo} elemento
de \(A\) também for elemento de \(B\). A notação \(A\subseteq B\)
significa que \(A\) é um \textbf{subconjunto} de \(B\).

Exemplo:

\(A=\{2,3,5\}\), \(B=\{1,2,3,4,5,6\}\) \(\Rightarrow\) \(A \subseteq B\)

\subsection{Operações sobre
Conjuntos}

\subsubsection{União}

Dados dois conjuntos \(A\) e \(B\), a \textbf{união} \(A\) com \(B\) é
definida por

\[A\cup B \Rightarrow x : x \in A \vee x \in B.\]

Exemplo:

\(A=\{1,3,7\}\)

\(B=\{2,4\}\)

\(A\cup B = \{1,2,3,4,7\}\)

\subsubsection{Intersecção}

Dados dois conjuntos \(A\) e \(B\), a \textbf{intersecção} \(A\) com
\(B\) é definida por

\[A\cap B \Rightarrow x : x \in A \land x \in B.\]

Exemplo:

\(A=\{1,2,3,4,7\}\)

\(B=\{2,4,5\}\)

\(A\cap B = \{2,4\}\)

\subsubsection{Diferença}

Dados dois conjuntos \(A\) e \(B\), a \textbf{diferença} de \(A\) com
\(B\) é definida por

\[A\backslash B \Rightarrow x : x \in A \land x \notin B.\]

Exemplo:

\(A=\{1,2,3,4,7\}\)

\(B=\{2,4,5\}\)

\(A\backslash B = \{1,3,7\}\)

\subsubsection{Diferença Simétrica}

Dados dois conjuntos \(A\) e \(B\), a \textbf{diferença simétrica} de
\(A\) com \(B\) é definida por

\[A\Delta B \Rightarrow (A \cup B) \backslash (A \cap B).\]

Exemplo:

\(A=\{1,2,3,4,7\}\)

\(B=\{2,4,5\}\)

\(A\backslash B = \{1,3,5,7\}\)

\subsubsection{Produto Cartesiano}

Dados dois conjuntos \(A\) e \(B\), o \textbf{produto cartesiano} é
definido por

\[A\times B = \{(x,y) : x \in A \lor y \in B\}\]

\((x,y)\) é uma \textbf{tupla} ordenada onde \(x\) pertence a \(A\) e
\(y\) a \(B\).

Exemplo: \(A=\{1,2,4\}\) \(B=\{2,3,4,7\}\)

\(A\times B = \{(1,2),(1,3),(1,4),(1,7),(2,2),(2,3),(2,4),(2,7),(4,2),(4,3),(4,4),(4,7)\}\)

\bigskip\noindent
\centerline{\bf Exercícios}

\paragraph{1.}~Para os conjuntos \(A = \{a, d, e, f, g\}\) e
\(B = \{a, b, c, e, h, i\}\), realize as seguintes operações:

\begin{enumerate}
\def\labelenumi{\alph{enumi}.}
\item
  \(A \cup B\)
\item
  \(A \cap B\)
\item
  \(A \backslash B\)
\item
  \(A \Delta B\)
\end{enumerate}

\paragraph{2.}~Dados os seguintes conjuntos: \[A=\{p,q,t,w\}\] \[ B = \{r,t,v\}\]
  \[C=\{p,s,t,u\}\]

\noindent Então determine:

\begin{enumerate}
\def\labelenumi{\alph{enumi}.}
	\begin{minipage}{.5\textwidth}
		\item \(B\cap C\)
		\item \(B\cap C\)
		\item  \(A\cup C\)
		\item \(|A\cup C|\)
		\item  \(A \cap B \cap C\)
	\end{minipage}
	\begin{minipage}{.5\textwidth}
		\item  \(B \backslash C\)
		\item  \((A\cup B)\cap C\)
		\item  \(A\times B\)
		\item  \(B\times C\)
	\end{minipage}
\end{enumerate}

\paragraph{3.}~Dados os seguintes conjuntos: \[A=\{2,4,5,6,7\}\] \[B=\{1,4,5,9\}\]
  \[C=\{x:x\in \mathbb{Z}\land 2\leq x<5\}\]

\noindent Então, determine:

\begin{enumerate}
\def\labelenumi{\alph{enumi}.}
	\begin{minipage}{.5\textwidth}
		\item  \(B\cup C\)
		\item  \(|B\cup C|\)
		\item  \(B\cap C\)
		\item  \(|B\cap C|\)
	\end{minipage}
	\begin{minipage}{.5\textwidth}
		\item  \(A\cup C\) f \(B\backslash A\)
		\item  \(C\backslash A\)
		\item  \((C\backslash A)\cap (A\backslash C)\)
		\item  \(A\times B\)
		\item  \(B\times C\)
	\end{minipage}
\end{enumerate}

\paragraph{4.}~Para os conjuntos \(A=\{1,2,3,4,5\}\) e \(B=\{4,5,6,7\}\), determine:

\begin{enumerate}
\def\labelenumi{\alph{enumi}.}
	\begin{minipage}{.5\textwidth}
		\item  \(A\cup B\)
		\item  \(A\cap B\)
		\item  \(A\backslash B\)
		\item  \(B\backslash A\)
	\end{minipage}
	\begin{minipage}{.5\textwidth}
		\item  \(A\Delta B\)
		\item  \(B\Delta A\)
		\item  \(A\times B\)
		\item  \(B\times A\)
	\end{minipage}
\end{enumerate}

\paragraph{5.}~Sejam \(A\) e \(B\) conjuntos e suponha que
  \(A\times B=\{(1,2),(1,3),(2,2),(2,3)\}\). Encontre \(A\cup B\),
  \(A\cap B\), \(A\backslash B\).


\paragraph{6.}~Suponha que \(A\) e \(B\) sejam conjuntos finitos. Dado que
  \(|A|=10\), \(|A\cup B|=15\) e \(|A\cap B|=3\), determine \(|B|\).

\subsection*{Referências}

\begin{itemize}
\item
  Clifford Stein, Robert L. Drysdale, Kenneth Bogart.
  \href{https://plataforma.bvirtual.com.br/Acervo/Publicacao/3824}{``Matemática
  Discreta para Ciência da Computação''}. Editora Pearson, 1$^a$ Edição,
  2013.
  \href{https://plataforma.bvirtual.com.br/Acervo/Publicacao/3824}{Biblioteca
  Virtual}
\item
  Edward R. Scheinerman.
  \href{https://www.amazon.com.br/Matem\%C3\%A1tica-Discreta-Introdu\%C3\%A7\%C3\%A3o-Edward-Scheinerman/dp/8522107963/}{``Matemática
  Discreta: Uma Introdução'', 2$^a$ edição}. Cengage Learning, 2011.
  (Existe uma versão mais recente: 3a. edição)
\end{itemize}

