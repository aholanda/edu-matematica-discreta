\def\T{V}
\def\F{F}

\section*{Fundamentos de Lógica}

\subsection*{Lógica Proposicional}

É a matemática de \(2\) (dois) valores Booleanos, \texttt{VERDADEIRO}
(\texttt{TRUE}, \texttt{T}, \texttt{\T} \texttt{1}) e \texttt{FALSO}
(\texttt{FALSE}, \texttt{F}, \texttt{\F}, \texttt{0}), e \(5\) (cinco) operadores.

\subsubsection*{Operadores da Lógica
Proposicional}

\begin{longtable}[]{@{}lccc@{}}\hline
& operador & nome & programação\\\hline
\endhead
1 & \(\land\) & conjunção & \texttt{AND}, \texttt{\&\&}\\
2 & \(\lor\) & disjunção & \texttt{OR},
\texttt{\textbackslash{}\textbar{}\textbackslash{}\textbar{}}\\
3 & \(\lnot\) & negação & \texttt{NOT}, \texttt{!}\\
4 & \(\implies\) & implicação &\\
5 & \(\equiv\) & equivalência & \texttt{==}\\\hline
\end{longtable}

    \textbf{Lógica Proposicional: Exemplos}

\begin{enumerate}
\def\labelenumi{\arabic{enumi}.}
\item
  \(x \land y\) é \texttt{VERDADEIRO} se \(x\) e \(y\) forem
  \texttt{VERDADEIRO};
\item
  \(x \lor y\) é \texttt{VERDADEIRO} se \(x\) ou \(y\) forem
  \texttt{VERDADEIRO} (ou ambos);
\item
  \(\lnot x\) é \texttt{VERDADEIRO} se \(x\) for \texttt{FALSO};
\item
  \(x \implies y\) é \texttt{VERDADEIRO} se \(x\) for \texttt{FALSO} e
  \(y\) for \texttt{VERDADEIRO} (ou ambos);
\item
  \(x\equiv y\) é \texttt{VERDADEIRO} se \(x\) e \(y\) forem
  \texttt{VERDADEIRO} ou ambos forem \texttt{FALSO}.
\end{enumerate}

    \textbf{Lógica Proposicional: Tabela-Verdade}

\begin{longtable}[]{@{}llllll@{}}
\(x\) & \(y\) & & & \(\land\) & \(\lor\)\tabularnewline
\emph{\texttt{\F}} & \emph{\texttt{\F}} & & & \texttt{\F} &
\texttt{\F}\tabularnewline
\emph{\texttt{\F}} & \emph{\texttt{\T}} & & & \texttt{\F} &
\texttt{\T}\tabularnewline
\emph{\texttt{\T}} & \emph{\texttt{\F}} & & & \texttt{\F} &
\texttt{\T}\tabularnewline
\emph{\texttt{\T}} & \emph{\texttt{\T}} & & & \texttt{\T} &
\texttt{\T}\tabularnewline
\end{longtable}

    \begin{longtable}[]{@{}llllll@{}}
\(x\) & \(y\) & & & \(\Rightarrow\) & \(\equiv\)\tabularnewline
\emph{\texttt{\F}} & \emph{\texttt{\F}} & & & \texttt{\T} &
\texttt{\T}\tabularnewline
\emph{\texttt{\F}} & \emph{\texttt{\T}} & & & \texttt{\T} &
\texttt{\F}\tabularnewline
\emph{\texttt{\T}} & \emph{\texttt{\F}} & & & \texttt{\F} &
\texttt{\F}\tabularnewline
\emph{\texttt{\T}} & \emph{\texttt{\T}} & & & \texttt{\T} &
\texttt{\T}\tabularnewline
\end{longtable}

    \begin{longtable}[]{@{}cc@{}}
\(x\) & \(\lnot x\)\tabularnewline
\texttt{\F} & \texttt{\T}\tabularnewline
\texttt{\T} & \texttt{\F}\tabularnewline
\end{longtable}

\subsection*{Lógica de Predicados (lógica de primeira
ordem)}

A Lógica de Predicados estende a Lógica Proposicional com a adição de
operadores que trabalham com conjuntos de elementos.

\textbf{Operadores}

\begin{longtable}[]{@{}cc@{}}
operador & operação\tabularnewline
\endhead
\(\forall\) & quantificação universal (para todo)\tabularnewline
\(\exists\) & quantificação existencial (existe)\tabularnewline
\end{longtable}

    \textbf{Exemplos}

\((\forall\ x \in R : S)\) - para todo \(x\) em \(R\), a fórmula \(S\) é
\texttt{VERDADEIRA};

\((\exists \ x \in R : S)\) - existe \(x\) em R, em que a fórmula \(S\)
é \texttt{VERDADEIRA}.

\(\forall\ x \in \{2, 3, 4, 5\} : x > 1\)

\(\exists\ x \in \{2, 3, 4, 5\} : x > 2\)

    \(\forall\ x \in N : x\ mod\ 2 = 0\) (é par?) \texttt{\F}

\(\exists\ x \in N : x\ mod\ 2 = 0\) (é par?) \texttt{V}

\(\exists\ x \in N : \sqrt{x} = 2\) \texttt{V}

\subsection*{Exercícios}

\begin{enumerate}
\def\labelenumi{\arabic{enumi}.}
\item
  Resolva as expressões a seguir para \((x,y)=(\T,\T)\), \((x,y)=(\T,\F)\),
  \((x,y)=(\F,\T)\) e \((x,y)=(\F,\F)\):
\end{enumerate}

\begin{itemize}
\item
  As expressões são sempre avaliadas da esquerda para a direita com
  exceção da existência de parênteses, por exemplo para
  \(x \lor (x \land y)\), primeiro avalia-se \((x \land y)\) e depois a
  disjunção \(\lor\);
\item
  \((x,y)=(\T,\F)\) significa que \(x=\T\) e \(y=\F\);
\item
  Se preferir, monte uma tabela-verdade para cada expressão.
\end{itemize}

\begin{enumerate}
\def\labelenumi{\alph{enumi}.}
\item
  \(x\land y \lor x\)
\item
  \(\neg (x\lor y \land x)\)
\item
  \(x \land y \implies \neg x\)
\item
  \((x \implies y) \equiv (x \lor y)\)
\item
  \((x \land y) \equiv (x \lor y)\)
\item
  \((\neg x \land y) \equiv (x \lor \neg y)\)
\item
  \(\neg(x \implies y) \equiv (x \land y)\)
\item
  \(x \lor (x \lor y) \land \neg(x \land \neg y)\)
\end{enumerate}

\begin{enumerate}
\def\labelenumi{\arabic{enumi}.}
\setcounter{enumi}{1}
\item
  Para o conjunto dos números inteiros \(Z\)
  (\(\ldots, -3, -2, -1, 0, 1, 2, 3, \ldots\)). Avalie as seguintes
  expressões:
\end{enumerate}

\begin{enumerate}
\def\labelenumi{\alph{enumi}.}
\item
  \(\forall x \in Z : x \text{ é par} \land\ x \text{ é impar}\)
\item
  \(\exists x \in Z : x \text{ é par} \land\ x \text{ é impar}\)
\item
  \(\forall x \in Z : x \text{ é par} \lor\ x \text{ é impar}\)
\item
  \(\exists x \in Z : x \text{ é par} \lor\ x \text{ é impar}\)
\item
  \(\forall x \in Z : x \text{ é primo}\)
\item
  \(\exists x \in Z : x \text{ é primo}\)
\item
  \(\neg(\forall x \in Z : x \text{ é primo} )\)
\item
  \(\forall x \in \{2,4,6,8\}, \forall y \in \{1,3,5,7\} : x + y = 11\)
\item
  \(\exists x \in \{2,4,6,8\}, \forall y \in \{1,3,5,7\} : x + y = 11\)
\item
  \(\exists x \in \{2,4,6,8\}, \exists y \in \{1,3,5,7\} : x + y = 11\)
\item
  \(\forall x \in \{0\}, \forall y \in \{1,3,5,7\} : xy = 0\)
\end{enumerate}

\begin{enumerate}
\def\labelenumi{\arabic{enumi}.}
\setcounter{enumi}{2}
\item
  {[}Scheinerman{]} Assinale como verdadeira (\T) ou false (\F) cada uma
  das expressões sobre inteiros a seguir:
\end{enumerate}

\begin{enumerate}
\def\labelenumi{\alph{enumi}.}
\item
  \(\forall x, \forall y : x + y = 0\)
\item
  \(\forall x, \exists y : x + y = 0\)
\item
  \(\exists x, \forall y : x + y = 0\)
\item
  \(\exists x, \exists y : x + y = 0\)
\item
  \(\forall x, \forall y : xy = 0\)
\item
  \(\forall x, \exists y : xy = 0\)
\item
  \(\exists x, \forall y : xy = 0\)
\item
  \(\exists x, \exists y : xy = 0\)
\end{enumerate}
