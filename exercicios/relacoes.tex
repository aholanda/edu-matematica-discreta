
%%% Local Variables: 
%%% mode: latex
%%% TeX-master: main.tex
%%% End: 

\def\<#1\>{\langle #1 \rangle}

\subsection{Relações}
\setcounter{exc}{0}

\ifnum1=2

\exercise{} Sejam $A=\{2,3,4,5\}$ e $B=\{3,4,5,6,10\}$. Para cada uma
das seguintes relações:
\label{ex:rel}
\begin{enumerate}[(1)]
\item explicite os pares ordenados da rela\c{c}\~ao;
\item fa\c{c}a a representação gr\'afica;
\item determine o domínio da defini\c{c}\~ao;
\item determine o conjunto imagem.
\end{enumerate}

\begin{enumerate}[(a)]
\item $R_1=\{\langle x,y\rangle\in A\times B : y \mbox{ é divisível
por } x\}$
\item $R_2=\{\langle x,y\rangle\in A\times B : x \mbox{ é par } \land
y \mbox{ é impar}\}$
\item $R_3=\{\langle x,y\rangle\in A\times B : x \mbox{ e } y \mbox{
sejam números primos}\}$
\item $R_4=\{\langle x,y\rangle\in A\times B : y/x=2\}$
\end{enumerate}

\exercise{} Para os mesmos conjuntos do exerc\'icio anterior, determine
o conjunto resultante das seguintes opera\c{c}\~oes:

\begin{enumerate}[(a)]
\item $A\cup B$
\item $A\cap B$
\item $A\backslash B$
\item $B\backslash A$
\item $A\Delta B$
\item $A\times B$
\end{enumerate}

\exercise{} Ache as rela\c{c}\~oes inversas $R^{-1}$ para as rela\c{c}\~oes
obtidas no exercício~1.

\fi

\subsection{Propriedades de uma endorrelação}

\exercise{} Determinar se as seguintes rela\c{c}\~oes s\~ao reflexiva,
anti-reflexiva, sim\'etrica, anti-sim\'etrica e transitiva:

\begin{enumerate}[(a)]
\item
$R_1=\{\tupla{1,2},\tupla{1,1},\tupla{2,2},\tupla{2,1},\tupla{3,3}\}$
\item
$R_2=\{\tupla{1,1},\tupla{2,2},\tupla{3,3},\tupla{1,2},\tupla{2,3}\}$
\item $R_3=\{\tupla{1,1},\tupla{1,2},\tupla{2,3},\tupla{3,1}\}$
\item $R_4 = A\times A$
\item $R_5 = \emptyset$
\end{enumerate}

\exercise{} [Antonio Alfredo Ferreira Loureiro] Para as relações a
seguir definidas no conjunto $\{1,2,3,4\}$, determinar se elas possuem
reflexividade, anti-reflexividade, simetria, anti-simetria,
transitividade e equivalência:

\begin{enumerate}[a)]
\item $R_1=\{\<2,2\>,\<2,3\>,\<2,4\>,\<3,2\>,\<3,3\>,\<3,4\>\}$ % N N N N S N
\item $R_2=\{\<1,1\>,\<1,2\>,\<2,1\>,\<2,2\>,\<3,3\>,\<4,4\>\}$ % S N S N S S
\item $R_3=\{\<2,4\>,\<4,2\>\}$ % N S S N N N 
\item $R_4=\{\<1,2\>,\<2,3\>,\<3,4\>\}$ % N S N S N N
\item $R_5=\{\<1,1\>,\<2,2\>,\<3,3\>,\<4,4\>\}$ % S N S S S S
\item $R_6=\{\<1,3\>,\<1,4\>,\<2,3\>,\<2,4\>,\<3,1\>,\<3,4\>\}$ % N S N N N N
\end{enumerate}

\begingroup\noindent{\bf Resposta:}
\begin{center}\footnotesize
\begin{tabular}{c|c|c|c|c|c|c}\hline
 &  reflexividade & anti-reflexividade& simetria& anti-simetria&transitividade & equivalência\\\hline
 a)& N&N&N&N&S&N\\\hline
 b)& S&N&S&N&S&S\\\hline
 c)& N&S&S&N&N&N\\\hline
 d)& N&S&N&S&N&N\\\hline
 e)& S&N&S&S&S&S\\\hline
 f)& N&S&N&N&N&N\\\hline
\end{tabular}
\end{center}
\endgroup

\begin{thebibliography}{1}
\bibitem[Menezes, 2009]{paulo2010} Paulo Blauth Menezes, Lara Vieira
Toscani, Javier García López. \emph{Aprendendo matemática discreta com
exercícios}. Editora Bookman, 2009.

\bibitem  [Scheinerman, 2003]{scheinerman2003} Edward R. Scheinerman. Matemática
  discreta: uma introdução. São Paulo: Thomson Learning, 2003.
\end{thebibliography}


