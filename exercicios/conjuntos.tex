
%%% Local Variables: 
%%% mode: latex
%%% TeX-master: "main.tex"
%%% End: 

\def\tupla#1{\langle #1\rangle}

Fontes: \cite{set:ed2010,set:blauth2010,set:edgard2008}

\subsection{Conjuntos}
\setcounter{exc}{0}

\exercise{}. Suponha o conjunto universo $S=\{p,q,r,s,t,u,v,w\}$,
bem como os seguintes conjuntos:
$$A=\{p,q,t,w\}$$
$$ B = \{r,t,v\}$$ 
$$C=\{p,s,t,u\}$$

Então determine:
\begin{enumerate}[(a)]
\item $B\cap C$
\item $A\cup C$
\item $\neg C$
\item $A \cap B \cap C$
\item $B \backslash C$
\item $\neg (A\cup B)$
\item $A \times B$
\item $(A\cup B)\cap C$
\end{enumerate}

% \exercise{}. Suponha o conjunto universo $S=\{0,1,2,3,4,5,6,7,8,9\}$,
% bem como os seguintes conjuntos:

% $$A=\{2,4,5,6,7\}$$
% $$B=\{1,4,5,9\}$$
% $$C=\{x:x\in \mathbb{Z}\land 2\leq x<5\}$$

% \noindent Então, determine:

% \begin{enumerate}[(a)]
% \item $B\cup C$
% \item $B\cap C$
% \item $A\cup C$
% \item $B\backslash A$
% \item $\neg B$
% \item $B\cap\neg B$
% \item $\neg(A\cap C)$
% \item $C\backslash A$
% \item $\neg (C\backslash A)\cap (A\backslash C)$
% \item $\neg (\neg C\cup A)$
% \item $A\times C$
% \end{enumerate}

\noindent $\land$ -- conector lógico ``e''.\\


\begin{thebibliography}{1}
\bibitem[Menezes, 2009]{set:blauth2010} Paulo Blauth Menezes, Lara Vieira
Toscani, Javier García López. \emph{Aprendendo matemática discreta com
exercícios}. Editora Bookman, 2009.

\bibitem  [Scheinerman, 2003]{set:ed2010} Edward R. Scheinerman. Matemática
  discreta: uma introdução. São Paulo: Thomson Learning, 2010.

\bibitem [Alencar-Filho,2008]{set:edgard2008} Edgard de Alencar Filho.
  {Inicia\c{c}\~ao A L\'ogica Matem\'atica}. Editora Nobel, 2008,
  21$^a$ edi\c{c}\~ao.

\end{thebibliography}


