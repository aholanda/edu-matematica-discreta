%%%%%%%%%%%%%%%%%%%%%%%%%%%%%%%% 1 %%%%%%%%%%%%%%%%%%%%%%%%%%%%%%%%
\lecturetitle{\course}{Teoria dos Conjuntos}

\frame{\maketitle}


\subsection{Quantificadores}

\begin{frame}{Quantificadores}{}

\begin{block}{Existe} 
\eumath{\exists x\in A}, afirmação sobre \eumath{x}.

\bigskip
Lê-se, existe \eumath{x} pertencente ao conjunto \eumath{A}, tal 
que a afirmação seja verdadeira. \eumath{x} é uma variável de 
referência.\\

\bigskip
Ex: \eumath{\exists x\in \mathbb{N}|} x é primo e par.
\end{block}

\begin{block}<2>{Para todo} 
\eumath{\forall x\in A}, afirmação sobre \eumath{x}.

\bigskip

Ex: \eumath{\forall x\in \mathbb{Z}}, x é ímpar ou par.
\end{block}  

\end{frame}

\begin{frame}{Combinação de quantificadores}

  \begin{enumerate}[<+-| alert@+>]
  \item Para todo x, existe um $y$ de modo que $x+y=0$,
    
    \[\forall x, \exists y | x+y=0;\]
  \item Existe um $x$, de modo que, para todo $y$, temos $x+y=0$;

    \[\exists x, \forall y | x+y=0.\]
  \end{enumerate}
  
\end{frame}

\begin{frame}{Exercícios}

  Descreva as afirmações abaixo usando os quantificadores e 
  sem se preocupar com sua veracidade.
 
  \begin{enumerate}
  \item Todo inteiro é primo.
  \item Há um inteiro que não é primo.
  \item Existe um inteiro cujo quadrado é $2$.
  \item Todos os inteiros são divisíveis por $5$.
  \item Algum inteiro é divisível por $7$.
  \item Para todo inteiro $x$, existe um inteiro $y$, de modo que 
    $xy=1$.
  \item Existem dois inteiros $x$ e $y$ de modo que $x/y=10$.
  \item Todos amam alguém alguma vez.
  \end{enumerate}

\end{frame}

\end{frame}
